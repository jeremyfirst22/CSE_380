\chapter{Conclusions\label{sec:conclusions}}
A modified finite element formulation is developed to simulate high-Reynolds number flows.  The scheme is an extension of the SUPG family augmented by a modified shock capturing operator which is required to eliminate spurious oscillations in the vicinity of shock waves.  The main features of this study concern improvements in numerical methodology for compressible Navier-Stokes simulation supported by accompanying verification simulations and an experimental validation study. 

 The verification test results for Mach~3 flow over a cylinder serves as a good test case for the effectiveness of the modified shock capturing operator (e.g. computed and theoretical jump values are in excellent agreement). The performance of the associated  transient, nonlinear, and mesh convergence was investigated.  The method was then validated by comparison to experimentally-measured quantities of interest such as surface pressure and heat transfer distributions.

The method is applicable to arbitrary unstructured discretizations, but the results shown here employ high-quality, structured grids.  The performance of the algorithm on unstructured meshes, including the influence of mesh quality on solution accuracy, is of interest and will be considered in future work.  This is a particularly important question as the ability to use hybrid-element unstructured meshes can greatly simplify the mesh generation process. Additional work will also examine how the specific choice of inviscid flux discretization (Equation~\eqref{eq:disc_F_expanded}) enhances the numerical stability of the method.

While only laminar, calorically perfect gases are considered in this work, the approach is expected to generalize directly to the case of turbulent and/or reacting flows.  Future work will extend the range of applicability of the finite element model by including state equations for gases in thermal equilibrium.  The effects of turbulence may be included through the typical Reynolds-Averaged Navier-Stokes approach by implementing suitable turbulence models.   Additionally, the highly localized shock waves and boundary layers which occur in this class of flows are well-suited for simulation with adaptive mesh refinement techniques, and such simulations will be the focus of future research.


