\chapter{Introduction}

The HyFlow Hypersonic Flow Simulation Toolkit is a suite of codes intended to be used in the design and analysis of entry vehicles at hypersonic conditions with a requisite thermal protection system.  The HyFlow software suite is a collaborative development effort between the National Aeronautics and Space Administration's Lyndon B.\ Johnson Space Center (NASA/JSC) and The University of Texas at Austin's Center for Predctive Engineering and Computational Sciences (UT/PECOS) group.

Hypersonic flows are characterized by exceptionally high temperatures, both in the flowfield external to an entry vehicle and even at the surface of the vehicle itself.  As a consequence, the flowfield external to the vehicle may be in a state of thermochemical nonequilibrium.  In such cases, solution of the flowfield requires not only satisfying the traditional Navier-Stokes equations with mass, momentum, and energy conservation -- but also determining the composition of the gas itself, and potentially even its internal energy distributions.  Such flows are said to be in \emph{thermochemical nonequilibrium} because neither their thermodynamics nor chemical composition may be considered to be in equilibrium.  In the HyFlow the hypersonic flowfield is approximated numerically using a number of mathematical models implemented in the \emph{Fully-Implicit Navier-Stokes} (\texttt{FIN-S}) flow solver.  \texttt{FIN-S} and its associated mathematical models are the subject of Chapter~\ref{chap:fin-s}.